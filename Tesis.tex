	% !TeX program = xelatex
	% !TeX spellcheck = es_ES
\documentclass[12pt,letterpaper]{book} 
\usepackage[utf8x]{inputenc}	% input encoding
\usepackage[spanish]{babel}		% language
\usepackage[T1]{fontenc}
\usepackage{fontspec}
\usepackage[scale=0.8]{roboto-mono}	% scaled roboto mono for texttt shape
\usepackage[expert,sfscaled,utopia]{mathdesign}
\usepackage[scaled=0.95]{classico}
\usepackage[round,elide]{natbib}		% bibliography with round parenthesis
	\bibliographystyle{authordate1}		%authordate1: initial caps; 3: author in smallcaps
\usepackage{ucs} 				% in conjunction with inputenc, to write in UTF-8
\usepackage{csquotes}			% for specifying quotes
	\MakeOuterQuote{"}
\usepackage{makeidx}			% to create analytical index
	\makeindex
\usepackage{amsmath}
%\usepackage{amsfonts}
%\usepackage{amssymb}
\usepackage{amsthm}				
	\newtheorem{definition}{Definici\'on}		% to define Definicion environment
	\newtheorem{theorem}{Teorema}				% to define Teorema environment
\usepackage{physics}			% for bra ket notation
\usepackage{multirow}			% for multiple rows in a table (used for titlepage)
\usepackage{longtable}			% for breakable tables
\usepackage{enumitem}			% to modify enumeration environments
	\setlist[description]{font=\normalfont\scshape}	% descriptions with small caps
	\renewcommand{\labelitemi}{\textbullet}	% itemize with bullet instead of blacksquare
\usepackage{lettrine} 			% for capital letters at the start of the chapter
% \usepackage{hanging}			% for hanging paragraphs in bibliography % removed for natbib
\usepackage{color}				% to use colors
\usepackage{xcolor}				% to define colors used for tables and titles
	\definecolor{blueish}{RGB}{4,122,255}
	\definecolor{aqua}{RGB}{0,100,150}
	\definecolor{deepblue}{RGB}{0,102,205}
	\definecolor{newgray}{RGB}{77,77,77}
	\definecolor{newblue}{RGB}{0,0,89}
	\definecolor{chapopcolor}{RGB}{0,100,150}
\usepackage{titletoc}			% for table of contents at beginning of chapters
\usepackage[explicit]{titlesec}	% for title formats
	%\titleformat{\chapter}[hang]{\sffamily\huge\bfseries\color{blueish}}{Capítulo \thechapter.}{0.5em}{}[]{#1.}
	\titleformat{name=\chapter,numberless}[hang]{}{}{0pt}{\sffamily\huge\bfseries\color{chapopcolor!80!black} #1}
		\titlespacing*{name=\chapter,numberless}{0pt}{0pt}{*10}     %% adjust 30 as per need
	\titleformat{name=\section}{}{}{0pt}{\sffamily\Large\bfseries\color{chapopcolor}\thesection.  #1}
	\titleformat{name=\section,numberless}{}{}{0pt}{\sffamily\large\bfseries\color{black}#1}
	\titleformat{name=\subsection}{}{}{0pt}{\sffamily\large\bfseries\color{chapopcolor!90}\thesubsection. #1}
	\titleformat{name=\subsection,numberless}{}{}{0pt}{\sffamily\large\bfseries\color{black}#1}
	\titleformat{name=\subsubsection}{}{}{0pt}{\sffamily\normalsize\bfseries\color{chapopcolor} \quad \thesubsubsection. #1}
	\titleformat{name=\subsubsection,numberless}{}{}{0pt}{\sffamily\normalsize\bfseries\color{chapopcolor}\qquad#1}
	\titleformat{name=\paragraph}[runin]{}{}{0pt}{\sffamily\normalsize\bfseries\color{black}#1}
		\titlespacing*{name=\paragraph}{0pt}{1ex}{1ex}
\usepackage{graphicx}			% builds upon the graph­ics package
	\graphicspath{ {images/} }	% the folder for images
\usepackage[left=2.5cm,right=2.2cm,top=2.7cm,bottom=2.5cm]{geometry}
\usepackage{enotez}	% to use endnotes instead of footnotes
	\setenotez{reset=true,backref=true,list-heading = \section*{#1}}
	\let\footnote=\endnote
\usepackage{tikz}				% for tikz images
	\usetikzlibrary{babel}
	\usetikzlibrary{arrows.meta,calc,positioning,shapes,arrows,decorations.pathmorphing}
\usepackage{pgfplots}			% for graph plotting
	\pgfplotsset{width=12cm,compat=newest}
	\def\axisdefaultwidth{360pt}
	\pgfplotsset{
	every axis/.append style = {thick},tick style = {thick,black},
	%
	% #1 = x, y, or z
	% #2 = the shift value
	/tikz/normal shift/.code 2 args = {%
		\pgftransformshift{%
			\pgfpointscale{#2}{\pgfplotspointouternormalvectorofticklabelaxis{#1}}%
		}%
	},%
	%
	range3frame/.style = {
		tick align        = outside,
		scaled ticks      = false,
		enlargelimits     = false,
		ticklabel shift   = {10pt},
		axis lines*       = left,
		line cap          = round,
		clip              = false,
		xtick style       = {normal shift={x}{10pt}},
		ytick style       = {normal shift={y}{10pt}},
		ztick style       = {normal shift={z}{10pt}},
		x axis line style = {normal shift={x}{10pt}},
		y axis line style = {normal shift={y}{10pt}},
		z axis line style = {normal shift={z}{10pt}},
			}
		}	
\usepackage{tcolorbox}			% for colored boxes
	\tcbuselibrary{breakable,skins}
\newtcolorbox{inicio}[1][]	% for boxes containing chapter ToC
{enhanced jigsaw,center,colframe=blue!50!black!50!white,colback=white,opacityback=0.7,drop fuzzy shadow=blue!50!black!60!white,boxrule=0.4pt}
\usepackage[breaklinks=true,
colorlinks=true,linkcolor=newblue,urlcolor=newblue,citecolor=newblue,anchorcolor=newblue,
bookmarks=true,pdfpagelayout=TwoPageRight]{hyperref}		% for hyperlinks
\usepackage{epigraph}			% for epigraphs
	\setlength\epigraphwidth{.7\textwidth}
	\setlength\epigraphrule{0pt}
	\renewcommand{\textflush}{flushepinormal}
% \usepackage[skip=0.3em,indent=1em]{parskip}
\renewcommand{\baselinestretch}{1.2}		% redefines space between lines
\raggedbottom % to allow ragged page bottoms
\setlength{\parskip}{0.3em}				% redefines space between paragraphs
%	\DeclareMathSymbol{\cfcond}{\mathrel}{symbolsC}{128} % Lewis’s ‘would’ counterfactual
%	\DeclareMathSymbol{\dcfcond}{\mathrel}{symbolsC}{132} % Lewis’s ‘might’ counterfactual
\newcommand{\counterfactual}{\ensuremath{%			to define Lewis's counterfactual
		\Box\kern-1.5pt
		\raise0.1pt\hbox{$\mathord{\rightarrow}$}}}

%%%%%% INICIO - FORMATO CHAPTER OPENER %%%%%% 
	\titleformat{name=\chapter}[display]%
	{\normalfont\hfuzz=\maxdimen}%
	{\color{gray}\raggedleft\fontsize{55}{2}\sffamily\bfseries\setlength{\baselineskip}{0pt}\thechapter}%%	FOR CHAP NUMBER
	{0.5pc}%
	{\color{chapopcolor!80!black}\centering\fontsize{30}{30}\sffamily\bfseries #1}%
	\titlespacing*{\chapter}{0pt}{*1}{*3}
%%%%%% FIN - FORMATO Capítulo %%%%%% 
%%%%%% BEGIN - TABLE OF CONTENTS FORMAT %%%%%%
	\contentsmargin{0cm}%
	%%%% CHAPTER IN TOC
	\titlecontents{chapter}[0pc]
	{\addvspace{18pt}\hypersetup{linkcolor=black} \sffamily}%
	{\bfseries\chaptertitlename~\thecontentslabel. \large}
	{\qquad\qquad \large}
	{\hypersetup{linkcolor=black}\normalsize\sffamily\dotfill \thecontentspage}[\addvspace{-2pt}]%
	%%%% SECTION IN TOC
	\titlecontents{section}[3.6pc]
	{\hypersetup{linkcolor=black}\itshape\addvspace{10pt}}
	{ } %{\contentslabel[\thecontentslabel]{2.4pc}}
	{}
	{\hypersetup{linkcolor=black}\upshape \dotfill\small\sffamily \thecontentspage}
	[]
	%%%% SUBSECTION IN TOC
	\titlecontents*{subsection}[4.4pc]
	{\hypersetup{linkcolor=black}\itshape\addvspace{1pt}\footnotesize}
	{}
	{}
	{\hypersetup{linkcolor=black} \footnotesize(\thecontentspage)}
	[ \textbullet\ ][]
	%%%% PALABRA -ÍNDICE-
\makeatletter
\renewcommand{\tableofcontents}{%
	\chapter*{%
		\vspace*{-20pt}%\p@}%
		\begin{tikzpicture}[remember picture, overlay]%
		\pgftext[right,x=15cm,y=0.2cm]{\Huge\bfseries\sffamily Índice}% \contentsname};%
		\end{tikzpicture}}%
	\@starttoc{toc}}
\makeatother

% \addto\captionsspanish{\renewcommand{\contentsname}{Contenidos}} % change ToC name with [spanish]babel

\author{Carlos Romero}
\title{Naturalización de la Metafísica Modal}


\begin{document}

\frontmatter

\begin{titlepage}
\begin{tcolorbox}[enhanced,spread=1pt,interior style={top color=white,bottom color=white}
%interior style={top color=white,bottom color=blue!10}
]
%\begin{minipage}[c][0.1\textheight][t]{1\textwidth}
	\vspace{3cm}
	\begin{center}
	\includegraphics[width=3cm, height=3cm]{unam}
% for comparison {\sffamily\Large \bfseries \scshape {Universidad Nacional Aut\'onoma de M\'exico}}\\
	\end{center}
%\end{minipage}
	\begin{minipage}[c][0.1\textheight][t]{1\textwidth}
		\begin{center}
			\vspace{.7cm}
			{\sffamily\Large \bfseries \scshape {Universidad Nacional Aut\'onoma de M\'exico}}\\
			\vspace{.2cm}
			{\sffamily\scshape Programa de Maestría y Doctorado en Filosofía}
		\end{center}
	\end{minipage}
	\begin{minipage}[c][0.6\textheight][t]{1\textwidth}
		\begin{center}
			\vspace{1.7cm}
			{\sffamily\LARGE\scshape\bfseries \color{aqua}{Naturalizaci\'{o}n de la Metaf\'{i}sica Modal}}\\
			\vspace{1.7cm}
			\makebox[5cm][c]{\sffamily TESIS}  \\%[8pt]
			{\sffamily que para optar por el grado de} \\[5pt]
			{\sffamily \large \textbf{Doctor en Filosof\'ia}}\\[40pt]            
			{\sffamily PRESENTA:}\\[5pt]
			{\sffamily \large \textbf{Carlos Alberto Romero Castillo}}\\
			\vspace{1cm}
				\begin{tabular}{l l l}
				{\sffamily Tutor:} & {\sffamily\scshape Dr. Elias Okon Gurvich} & {\sffamily\small IIFs, UNAM}\\
				\multirow{2}{*}{\sffamily Comit\'e:} & {\sffamily\scshape Dr. Alessandro Torza} & {\sffamily\small IIFs, UNAM}\\
				& {\sffamily\scshape Dra. Lourdes Valdivia Dounce} & {\sffamily\small FFyL, UNAM}\\
				\end{tabular}	\\
			\vspace{0.5cm}
			{\sffamily Ciudad de M\'exico,}{ }{\sffamily Octubre de 2020} 	
		\end{center}
	\end{minipage}
\end{tcolorbox}
\end{titlepage}


\pagestyle{plain}

	\tableofcontents

\newpage

 \epigraph{This metaphysical issue is taken as the fundamental one for the realism debate by van Fraassen, and he is right: \textit{metaphysics in general, and modal metaphysics in particular, are the crux of scientific realism}.}{---J. Ladyman}%, `Scientific Realism Again'}
 
 
 \epigraph{\textit{One's attitude toward modalities has a profound effect on one's whole theory of science}. Actualists, including actual realists, must hold that the aim of science is primarily to describe the actual history of the world. For modalists, including modal empiricists, the aim is to describe the structure of physical possibility (or propensity) and necessity. The actual history is just that one possibility that happened to be realized. This difference in aims is connected with profound differences in how one understands diverse scientific activities such as causal attribution, explanation, and experimental design.}{---R. Giere}%, `Constructive Realism'}


 \newpage
 

\chapter*{Agradecimientos}
\addcontentsline{toc}{chapter}{Agradecimientos}
 \input{capitulos/agradecimientos}


\newpage 
 

\chapter*{Resumen}
\addcontentsline{toc}{chapter}{Resumen}
 \input{capitulos/resumen}


\chapter*{English Abstract}
\addcontentsline{toc}{chapter}{Abstract in English}
 \input{capitulos/abstracteng}


\mainmatter


\chapter{Introducción}\label{ch:Introduccion}
 \input{capitulos/01_introduccion}


\chapter{Contra la \textit{modalidad metafísica}}\label{ch:ConceptoModalidadMet}
 \input{capitulos/02_modalidad_metafisica}


\chapter{?`Es posible naturalizar a la metafísica modal?}\label{ch:ProyectoNaturalizaciónModalidad}
 \input{capitulos/03_naturalizacion_modalidad}
 
 
\chapter{Espacios de posibilidad e ideología modal en las ciencias}\label{chap:EspaciosPosibilidad}
 \input{capitulos/04_espacios_posibilidad}


\chapter{Una metafísica naturalizada de la modalidad}\label{chap:RAMS}
 \input{capitulos/05_realismo_estructura_modal}


\chapter{La lógica modal naturalizada}\label{ch:LogicaModalNatural}
 \input{capitulos/06_logica_modal_naturalizada}


\chapter{Conclusiones}
 \input{capitulos/07_conclusiones}


\backmatter


% \chapter*{Referencias bibiliográficas}
\addcontentsline{toc}{chapter}{Bibliografía}
% \input{capitulos/referencias}
\bibliography{bibfile1}


% \appendix

	%\cleardoublepage
% \addcontentsline{toc}{chapter}{Índice analítico}
% \renewcommand\indexname{Índice analítico}
% \printindex

\end{document}
